\chapter{Summary, Conclusions, and Recommendations}

This chapter provides a summary of the study, presents the conclusions drawn from the results, and outlines recommendations for future work.

\section{Summary}

The primary goal of this study was to develop resources and models to support automatic speech recognition (ASR) for the Akeanon language. To achieve this, the following tasks were accomplished:
\begin{itemize}
    \item A text corpus consisting of approximately 24,000 verified Akeanon words was constructed, covering various root words, derivations, and inflections.
    \item Additional translations of the Swadesh 207-word list and SIL International's word list were created for five major Akeanon dialects.
    \item A speech corpus consisting of 100 voice recordings from various speakers was collected to provide training and evaluation material.
    \item Monophone and triphone acoustic models were developed and evaluated using a ten-fold cross-validation approach.
\end{itemize}

The constructed corpora and ASR models lay the groundwork for future research and applications in Akeanon language technology.

\section{Conclusions}

Based on the results of the study, the following conclusions were drawn:
\begin{itemize}
    \item The creation of a diversified and verified text corpus greatly contributed to the linguistic resources available for the Akeanon language.
    \item The collected speech data provided sufficient variability in pronunciation and intonation, which is important for robust acoustic modeling.
    \item The trained monophone and triphone models achieved promising word recognition accuracy, demonstrating the feasibility of building an ASR system for Akeanon using the developed corpora.
\end{itemize}

Overall, the study successfully demonstrated initial steps toward enabling ASR capabilities for an underrepresented Philippine language.

\section{Recommendations}

In light of the findings and limitations of the study, the following recommendations are proposed for future research and development:
\begin{itemize}
    \item Expand the text and speech corpora by including more dialects, larger vocabularies, and a greater variety of speakers to improve model generalization.
    \item Explore advanced modeling techniques such as deep neural networks (DNNs) and end-to-end ASR systems for improved recognition performance.
    \item Conduct further experiments involving larger datasets and alternative language modeling approaches to enhance recognition accuracy.
    \item Investigate the development of tools and applications that can utilize the constructed corpora and ASR models for educational or cultural preservation purposes.
\end{itemize}

Continued development in these areas will contribute to the broader goal of promoting and preserving the Akeanon language in the digital age.


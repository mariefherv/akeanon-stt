% Filename : abstract.tex
\vspace*{-5em} % or fine-tune as needed
\begin{abstract}
\noindent
\justifying

Akeanon, a language spoken in Aklan, Philippines, is classified as a low-resource language (LRL) due to its limited linguistic resources and lack of digital integration. This research aimed to develop and establish a comprehensive text and speech corpus and build models as a foundation for an automatic speech recognition (ASR) system for the Akeanon language. A total of approximately 24,500 words were compiled for the text corpus. Data collection included compiling word lists for Akeanon based on the Swadesh 207 list and extracting text from both online and non-digital resources. After compiling the word lists, the entire collection was annotated accordingly, including phonetic transcriptions, in preparation for building and training the models. For the development of the speech corpus, more than 1,000 Akeanon words and the Swadesh list of different Akeanon dialects were voice recorded. Recordings were collected from fifty native speakers for the five random sets and ten native speakers for each Swadesh word list translated into their respective dialects, ensuring variation in gender, age, and dialect. The development of the speech and text corpus was overseen and validated by a linguistics expert and native speakers of the language. Monophone and triphone acoustic models were built and trained using Kaldi with the newly developed corpus for initial results. This study contributed to the preservation and digital inclusiveness of the Akeanon language and laid the groundwork for future efforts in developing an ASR system for the language.

\vspace{1em} % Space before keywords (inside abstract)
\noindent\textbf{Keywords:} Language resources, Natural language processing (NLP), Speech recognition, Philippine languages, Aklan, Aklanon, Akeanon, Language corpus, Low-resource languages (LRL)

\end{abstract}

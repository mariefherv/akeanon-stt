%   Filename    : abstract.tex 
\begin{abstract}
	
This special problem aims to develop and establish a speech and text corpus for the Akeanon language. Akeanon language, one of the Philippines local languages,  remains underrepresented in modern  Natural Language Processing (NLP) and speech recognition systems even Speech-to-Text (STT) technology has advanced significantly. Due to limited linguistic resources, Akeanon, an Austronesian language spoken in Aklan, Philippines, is classified as a low-resource language (LRL) due to the scarcity of language resources available for its study and the lack of digital integration. This research aims to develop a comprehensive text and speech corpus for standardized Akeanon language. By utilizing sources like the Swadesh word list, available online recordings, and dictionary references, the study gathers, transcribes, and phonologically encodes Akeanon words and their corresponding audio. To ensure diverse representation, a total of fifty native speakers with varying gender and age, contributed to the speech corpus. Normalization and noise reduction techniques for preprocessing are used to improve the overall quality of the audio recordings. Validation by linguistic experts ensures accuracy and reliability of the corpus.

The resulting Speech-to-Text corpus acts as an important resource for developing Automatic Speech Recognition (ASR) system for Akeanon language, facilitating its integration into NLP technologies. By addressing the challenges associated with LRL development, this study contributes to the preservation of local language, promotes digital inclusiveness, and increases accessibility to those who speak Akeanon. This study will also enhance STT technology for languages with limited resources. The insights obtained from tackling the problems associated with developing STT with low-resource language contribute to the broader spectrum of computational linguistics and highlight the importance of preserving indigenous language in the modern digital world.

\emph{{\color{red} To be followed by concrete result.}}
% From 150 to 200 words of short, direct and complete sentences, the abstract 
% should be informative enough to serve as a substitute for reading the entire SP document 
% itself.  It states the rationale and the objectives of the research.  
% In the final Special Problem  document (i.e., the document you'll submit for your final defense), the  abstract should also contain a description of your research results, findings,  and contribution(s).

%  Do not put citations or quotes in the abract.

\begin{flushleft}
\begin{tabular}{lp{4.25in}}
\hspace{-0.5em}\textbf{Keywords:}\hspace{0.25em} & Language resources, Natural language processing (NLP), Speech recognition, Philippine languages, Aklan, Akeanon, Language corpus, Low-resource languages (LRL)
\end{tabular}
\end{flushleft}
\end{abstract}

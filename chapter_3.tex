%   Filename    : chapter_4.tex 
\chapter{Research Methodology}
This chapter discusses the methodology used to develop the text and speech corpus for the Akeanon language, as well as building, training, and testing a model to generate initial results. The chapter will be divided into two main parts: Research activities and the calendar of activities for this special problem.

\section{RESEARCH ACTIVITIES}
Figure~\ref{fig:flowchart} shows the general overview of the methodology for the development of an ASR system for the Akeanon language.

\begin{figure}[h!]
	\centering
	\includegraphics[width=\textwidth]{./figures/flowchart.png}
   \caption{Research Methodology}
	\label{fig:flowchart}
\end{figure}

\subsection{Data Collection}

\textbf{Collating Pre-existing Online Resources}

For the data collection, the researchers will make use of existing online resources from the website, Bible.com (see Appendix A for permission to use their resources). These resources include recordings and transcriptions of the Akeanon translations of the multiple books and chapters of the Bible.

\textbf{Gathering, Encoding, and Digitization of Non-Digital Resources}

The researchers will gather different Akeanon-based resources and text available at Kalibo Municipal library, to which include a dictionaries and thesaurus in Akeanon, songs, fables and tales, poems, and different collections of Akeanon text. The gathered resources will then be manually encoded and converted into digital format. For dictionaries and thesaurus, it will be encoded and organized in a way that can be conveniently parsed for annotations. The Akeanon texts and literary pieces will be encoded and stored in plain text for future analysis.


\textbf{Compiling Akeanon Words}

The researchers will collect equivalent Akeanon words based off on the Swadesh 207 word-list, the Aklanon to English Dictionary by \citeA{Reyes:1969}, A Thesaurus in Aklanon by \citeA{Pastrana:2012}, and Diksyunaryong Akeanon-English-Filipino by \citeA{Belayro:2015} as references. All Akeanon words that can be found in all the collected and encoded resources will also be taken into account, including the collated pre-existing online resources. In addition, the different Akeanon dialects, namely Bukidnon, Buruangganon, Malaynon, and Nabasnon, will also be taken into consideration where the researchers will tap native speakers for each dialect and build on the Swadesh list as a starting point. The goal is to have at least 20000 Akeanon words in the collection, including the inflections, derivations, and dialect variations.

% \textbf{Phonetic Transcription}

% After compiling the Akeanon word lists, their phonetic transcription will also be encoded, using the work of \shortciteA{Rentillo:2022} as reference for Akeanon phonology. Audio transcriptions retrieved from the Bible.com will also have their phonetic transcriptions be encoded. For the phonetic transcription the researchers will seek assistance from Ms. Hazel Cipriano, a linguist who is also a native speaker of the language.

\textbf{Extraction and Storing}

After encoding and organizing the datasets across different sources accordingly, the data will be extracted and stored in a central database for the entire word collection. To ensure uniformity among various data sources, a word is stored in the following object format:

\begin{lstlisting}[language=json, caption=Object structure for storing a word, breaklines=true]
   {
   "word": "Akeanon word",
   "attributes": {
      "en_trans": ["English translation", "Other translation"],
      "type": "root", // Types: "root"/"inflection"/"derivation",
      "variation": "Standard" // Specify dialect/variation
      "pos": "noun", // Part-of-speech tag
      "stem": null,  // For "root" type, this is null.
      "affixes": { // If it is an inflection or derivation
         "prefixes": ["prefix1", "prefix2"],
         "suffixes": ["suffix1", "suffix2"],
         "infixes": ["infix1"]
      },
      "inflections": ["inflection1", "inflection2"],  
      "derivations": ["derived_word1", "derived_word2"],
      "synonyms": ["synonym1", "synonym2"],
      "antonyms": ["antonym1", "antonym2"],
      "examples": [
         {
         "sentence": "Example sentence using the word.",
         "translation": "Translated sentence."
         }
      ],
      "frequency": 123, // Frequency count across all sources,
      "source": "Source where the word was derived",
      "notes": "Additional notes"
   }}

\end{lstlisting}

For the database, the researchers will use Prisma Postgres for storage to efficiently manage complex relationships between the word objects and provide room for scalability as the corpus expands.

\textbf{Word Selection for Speech Corpus}

For building the speech corpus, the researchers will prioritize words from the Swadesh 207 list for the voice recordings. Additionally, to complete the selection of 1000 words that will be prepared for speakers to read, the researchers will collaborate with a linguistic expert during the selection to ensure that the word-lists will be phonetically balanced.

In preparation, the researchers will coordinate with  Ms. Hazel Cipriano, a linguistic expert and a native speaker of the language, in helping identify the possible phonemes and graphemes for the Akeanon language in consideration of the work of \shortciteA{Rentillo:2022} as well as reference for Akeanon phonology. After identifying the phonemes and graphemes for the language, the researchers will manually generate a word list that covers all the identified phonemes and common combinations. The word list will then be validated by the linguistic expert and as well as, with the aid of Dr. John E. Barrios, a native speaker of Akeanon and has numerous works featuring the said language, ensuring wide and even phoneme coverage and that the distribution of phonemes aligns with natural speech patterns.

\textbf{Voice Recording}

A total of 50 native speakers of the Akeanon language, including its dialects, will be gathered for the recording of the generated word list. The 1000 words list will be divided into five sets, with each containing 200 words that are unique to that set. Each set will have 10 speakers for the recording. The researchers have also sought a collaboration with Aklan State University (ASU) - College of Teacher Education for the selection of speakers, with Dr. John Orbista as the primary contact. The speakers will be of varying gender, age, and spoken dialect to ensure diversity. Table~\ref{tab:native_speakers} shows the categories of native speakers. For the audio recording, the microphone that will be used will be Shure SM58 (dynamic, cardiod pick-up pattern) with a Focusrite Scarlett 2i2 audio interface, having Adobe Audition 2021 as the recording software. In case of unavailability, recording over a smartphone with noise-cancelling headphones would suffice. The audio file will be named in the following convention: \texttt{\textless gender\textgreater\_\textless age\_group\textgreater\_\textless spoken\_dialect\textgreater\_\textless speaker\_number\textgreater}.

\begin{table}[ht]
   \centering
   \caption{Categories of Native Speakers} \vspace{0.25em}
   \label{tab:native_speakers}
   % Adjust row height and column padding
   \renewcommand{\arraystretch}{1.5} % Increase row height
   \setlength{\tabcolsep}{10pt} % Increase horizontal padding

\begin{tabular}{|c|p{2in}|} \hline
   \centering Category & Subcategories \\ \hline
   Sex & Male \\ 
   & Female \\ 
   \hline
   Age Group & 
   12-15 \\ 
   & 16-30 \\ 
   & 31-45 \\ 
   & 46-65 \\ \hline
   Spoken Dialect & 
   Standard Akeanon \\ 
   & Bukidnon \\ 
   & Buruangganon \\ 
   & Malaynon \\ 
   & Nabasnon \\ \hline
\end{tabular}
\end{table}

\textbf{Ethical Considerations}

During the gathering of the different Akeanon-based resources and text, the researchers have sought consent from the respective authors to use their works, in respect to intellectual property rights. A notable example was when the researchers have reached out to Dr. David Zorc, one of the authors of the Aklanon-to-English Dictionary. Not only who he granted the researchers permission to use the dictionary, but he has also generously provided a soft copy of an existing database containing all the contents of the dictionary to the researchers. This database was shared courtesy of Dr. Jarrette Allen, a linguist who has also worked on the Akeanon language.

At the beginning of their session for the voice recordings, participants will be provided with a consent form containing information relevant to the study. This consent form will serve as a formal acknowledgment of the participant's voluntary involvement and understanding of the study's objectives, procedures, and potential risks. The form will clearly explain the purpose of the research, how the data will be used, and the steps taken to ensure confidentiality and anonymity. Participants will be informed that they can withdraw from the study at any time without penalty. Additionally, the consent form will detail the nature of the voice recordings and the storage of their data. Participants will also be made aware that their voices may be used for research analysis but will not be associated with their personal identities.

For minor participants, additional ethical measures will be implemented. A separate Parental/Guardian Consent Form will be provided, which outlines the same key information regarding the study, along with specific assurances about the protection of the minor’s privacy and confidentiality. This form will seek explicit permission from the parent or guardian before the minor is allowed to participate. Parents or guardians will also be given the opportunity to ask questions and will be assured that their child’s participation is entirely voluntary. Furthermore, minors will be asked to provide assent—a simplified acknowledgment that they understand the study and agree to participate. Both the parent/guardian consent and the minor's assent will be required before participation can proceed. Throughout the study, the rights and welfare of minor participants will be prioritized, and measures will be taken to ensure their comfort and safety.

\subsection{Preprocessing}
For preprocessing audio files, Adobe Audition 2021 will be used for recording and audio processing, which will include normalization and noise reduction of the recorded audio. The cleaned up audio files will then be exported in a WAV format.

\subsection{Speech and Text Corpus Development}
The previous steps will set as a precedent for the speech and text corpus development. The development of the text corpus for the Akeanon language involves creating a comprehensive collection of audio transcriptions from pre-existing online resources, dictionary-based word list, and Akeanon literature and text. For the speech corpus, the collated audio transcriptions and word list will be mapped with their corresponding voice recordings and will be annotated accordingly.

\subsection{Validation}

To validate the text corpus, the researchers will coordinate with native speakers and language experts such as Dr. Barrios, to ensure the accuracy of the spelling and grammar. The researchers will also coordinate with Ms. Cipriano again to aid the researchers in validating the accuracy and consistency of the tokenization and annotations of the developed text corpus.

For the speech corpus validation, the researchers will again rely on native speakers, along with Dr. Barrios, to ensure the accuracy of the transcriptions. The transcription accuracy will be further verified by comparing the transcriptions to the spoken content and ensuring consistency across the entire corpus.

\subsection{Building and Training A Model}
To generate initial results for an ASR system, the researchers will build, train, and test a model using the Kaldi toolkit. Similar to the methods employed by \shortciteA{Panizales:2023}, a ten-fold cross-validation scheme will be used for training and testing the data, with eight folds used for training and two folds used for validation. Finally, the researchers have decided to use a DNN (Deep Neural Network) model for training.

\section{Calendar of Activities}

% A Gantt chart showing the schedule of the activities should be included as a table. For example:

Table \ref{tab:timetableactivities} shows a Gantt chart of the activities.  Each bullet represents approximately
one week worth of activity.

%
%  the following commands will be used for filling up the bullets in the Gantt chart
%
\newcommand{\weekone}{\textbullet}
\newcommand{\weektwo}{\textbullet \textbullet}
\newcommand{\weekthree}{\textbullet \textbullet \textbullet}
\newcommand{\weekfour}{\textbullet \textbullet \textbullet \textbullet}

%
%  alternative to bullet is a star 
%
\begin{comment}
   \newcommand{\weekone}{$\star$}
   \newcommand{\weektwo}{$\star \star$}
   \newcommand{\weekthree}{$\star \star \star$}
   \newcommand{\weekfour}{$\star \star \star \star$ }
\end{comment}


\begin{table}[ht]   %t means place on top, replace with b if you want to place at the bottom
\centering
\caption{Timetable of Activities} \vspace{0.25em}
\makebox[\textwidth]{
\resizebox{1.2\textwidth}{!}{
\begin{tabular}{|p{2in}|c|c|c|c|c|c|c|c|c|} \hline
\centering Activities (A.Y. 2024-2025) 														& Sep & Oct  & Nov & Dec & Jan & Feb & Mar & Apr & May \\ \hline
Brainstorming and Selection of Topic      													& \weekone & \weektwo &  &  &  &  &  &  & \\ \hline
Drafting and Finalization of Chapter 1 - Introduction 										&  & \weektwo & \weekone &  &  &  &  &  & \\ \hline
Drafting and Finalization of Chapter 2 - Review of Related Literature      					&   &  & \weekthree &  &  &  &  &  &  \\ \hline
Preparation of Letters; UPVREB and Establish Communication for Collaboration     			&   & \weekone &  &  &  &  &  &  &  \\ \hline
Drafting and Finalization of Chapter 3 - Methodologies      								&   &  & \weekone & \weektwo &  &  &  &  &   \\ \hline
Proposal Document Creation in LaTex 														&   &  & \weekone & \weekone &  &  &  &  &  \\ \hline
Proposal Presentation 																		&   &  &  & \weekone &  &  &  &  & \\ \hline
Data Gathering 																				&   &  &  & \weekthree & \weekthree  &  &  &  & \\ \hline
Preprocessing 																				&   &  &  &  &  \weekone  & \weekthree &  &  &  \\ \hline
Drafting and Finalization of Chapter 4 - Results and Discussion 							&   &  &  &  &  & \weekone & \weektwo &  &  \\ \hline
Corpus Development and Validation															&   &  &  &  &  &  & \weekthree & \weekthree &  \\ \hline
Drafting and Finalization of Chapter 5 - Summary and Recommendation							&   &  &  &  &  &  &  & \weektwo & \weektwo \\ \hline
Drafting and Finalization of SP Defense														&   &  &  &  &  &  &  &  & \weekone \\ \hline
SP Defense																					&   &  &  &  &  &  &  &  & \weekone \\ \hline
\end{tabular}
}}
\label{tab:timetableactivities}
\end{table}

